\documentclass{article}
\usepackage{epsfig}
\usepackage{hyperref}
\renewcommand{\baselinestretch}{1}
\setlength{\textheight}{9in}
\setlength{\textwidth}{6.5in}
\setlength{\headheight}{0in}
\setlength{\headsep}{0in}
\setlength{\topmargin}{0in}
\setlength{\oddsidemargin}{0in}
\setlength{\evensidemargin}{0in}
\setlength{\parindent}{.3in}
\begin{document}

\leftline{Pat Q.~Student}
\leftline{AME 20231}
\leftline{9 February 2021}

\medskip
This is a sample file in the text formatter \LaTeX.
I require you to use it for the following reasons:

\begin{itemize}

\item{It produces the best output of text, figures,
      and equations of any
      program I've seen.}

\item{It is machine-independent. It runs on Linux, Macintosh (see {\tt TeXShop}), and Windows (see {\tt MiKTeX}) machines.  There are web-based versions, \href{https://www.overleaf.com}{\tt https://www.overleaf.com}.
     You can e-mail {\tt ASCII} text versions of most relevant files.}

\item{It is the tool of choice for many research
     scientists and engineers.
     Many journals accept 
     \LaTeX~ submissions, and many books
     are written in \LaTeX.}

\end{itemize}
\medskip
Some basic instructions are given next.
Put your text in here.  You can be a little sloppy    about
spacing.  It adjusts the text to look good.
{\small You can make the text smaller.}
{\tiny You can make the text tiny.}

Skip a line for a new paragraph.   
You can use italics ({\em e.g.} {\em  Thermodynamics is everywhere}) or {\bf bold}.
Greek letters are a snap: $\Psi$, $\psi$,
$\Phi$, $\phi$.  Equations within text are easy---
A well known Maxwell thermodynamic relation is
$\left.{\partial T \over \partial P}\right|_{s} = 
\left.{\partial v \over \partial s}\right|_{P}$.
You can also set aside equations like so:
\begin{eqnarray}
du &=& T\ ds -P\ dv, \qquad \mbox{first law.}\label{fl}\\
ds &\ge& {\delta q \over T}.\qquad  \qquad \mbox{second law.} \label{sl}
\end{eqnarray}
Eq.~(\ref{fl}) is the first law.
Eq.~(\ref{sl}) is the second law.
References\footnote{Lamport, L., 1986, {\em \LaTeX: User's Guide \& Reference Manual},
    Addison-Wesley: Reading, Massachusetts.}
are available. 
If you have an postscript file, say {\tt sample.figure.eps}, in the same local directory,
you can insert the file as a figure.  Figure \ref{sample}, below, plots an isotherm for air modeled as an ideal gas. 
\begin{figure}[ht]
\epsfxsize=2.5in
\centerline{\epsffile{sample.figure.eps}}
\caption{Sample figure plotting $T=300~{\rm K}$ isotherm for air when modeled as an ideal gas.}
\label{sample}
\end{figure}

\medskip
\leftline{\em Running \LaTeX}
\medskip

You can create a \LaTeX~ file with any text editor ({\tt vi}, {\tt emacs}, {\tt gedit}, 
etc.). 
To get a document, you need to run the \LaTeX~ application
on the text file.  The text file must have the suffix ``{\tt .tex}''
On a Linux cluster machine, this is done via the command

\medskip
{\tt latex file.tex}

\medskip
\noindent
This generates three files: {\tt file.dvi}, {\tt file.aux},
and {\tt file.log}.  The most important is {\tt file.dvi}. 

\medskip
\noindent
The finished product can be previewed in the following way.
Execute the commands:

\medskip

{\tt dvipdf file.dvi}\hspace{1.9in}{\em Linux System}

\medskip
\noindent
This command generates {\tt file.pdf}.  
Alternatively, you can use {\tt TeXShop} on a Macintosh or {\tt MiKTeX} on a Windows-based machine. {\em Another very good and modern option is the web-based} \href{https://www.overleaf.com}{\tt https://www.overleaf.com}.
The {\tt .tex} file must have a closing statement as
below.


\end{document}
